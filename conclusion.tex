Collins researchers and product engineers are exploring many applications of machine learning.  In some cases, the goal is to implement new aircraft functionality using the unique capabilities of ML.  In other cases, the goal is to use a neural network to create a more time- or memory-efficient implementation of an existing function.  

A variety of assurance technologies have been developed and investigated on the DARPA Assured Autonomy program and related efforts.  These include new approaches for testing and completeness metrics, formal analysis of neural networks, input domain shift assessment, and run-time monitoring and enforcement architectures.
For each selected ML application, we determined one or more AA technologies that can be used to satisfy the relevant certification objectives.  We evaluated the effectiveness of the technologies and the evidence produced.

% What is our overall impression of these tools?
In general, we believe all the approaches covered in this paper can play a role in assuring low-complexity, low-DAL products.  The manifold-based test generation and property inference tools have not yet reached the necessary maturity levels to be used on programs of record, but their underlying methodologies show promise.  
We plan to share these findings with certification authorities to obtain feedback.
