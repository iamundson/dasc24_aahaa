We have evaluated the tools on a variety of ML-based systems.  In some cases, the system used ML to implement new aircraft functionality.  In other cases, the goal is to use a neural network (NN) to create a more time- or memory-efficient implementation of an existing function.  ML-based functions used for evaluation include:

\noindent{\bf Remaining Useful Life (RUL)~\cite{rul}}:  A convolutional neural network (CNN) uses vibration measurements from rotating equipment to estimate time until maintenance or replacement.  A public version (with training data and requirements to be verified) was made available as a benchmark for the 2022 VNN-COMP.


\noindent{\bf Recommended Cruise Level (RCL)}:  Computes time and fuel optimal altitude as a recommendation to the pilot, replacing a complex optimization calculation and saving CPU time. The NN is a fully-connected, feed-forward, rectified linear unit (ReLU) NN. It has 5 inputs and 2 outputs (i.e., time and fuel cost). It has 5 hidden layers, each with 10 neurons.


\noindent{\bf Fuel Quantity Measurement (FQM)}:  Computes fuel mass based on sensor measurements, replacing less-accurate or table-based implementations.  A NN is trained to invert a function that computes sensor measurements from fuel mass, fuel tank geometry, and aircraft orientation.  The NN is a fully-connected, feed-forward NN. It has 6 inputs: 3 pressure sensor signals and 3 acceleration signals. It has 1 output: fuel mass. The NN has 1 hidden layer consisting of 50 neurons. It uses the \emph{tanh} activation function. Inputs and outputs are normalized to [-1, 1] based on the training data range.


\noindent{\bf Runway Overrun Protection (ROP)}:  Estimates aircraft landing distance based on weight, speed, weather conditions, runway slope, and other parameters.  The NN is a fully-connected, feed-forward DNN. It has 10 inputs and 3 outputs. The 3 outputs represent landing distances with different brake settings. The NN has 2 hidden layers with 40 neurons each. It uses the \emph{tanh} activation function. Inputs and outputs are normalized to [-1, 1] based on the training data range.


\noindent{\bf Flight Trajectory Optimization (FTO)}: A neural network-based implementation of an optimized trajectory function, such as the A* algorithm and derivatives, to reduce computation time in a set of complex flight and weather conditions. The NN is a fully-connected, feed-forward DNN. Its input layer has 7 neurons, which model aircraft relative position to assigned flight altitude limit, distance to weather/threat, and relative velocity. It has 3 hidden layers with 35, 70, and 70 neurons, respectively. Its output layer has 5 neurons representing the next flight direction command: up, down, right, left, and straight. It uses the \textit{ReLU} activation function. 


